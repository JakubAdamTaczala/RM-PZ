\section{Zarządzanie projektem}
\label{sec:zarzadzanie_projektem} %definicja etykiety rozdzialu
Gotowy produkt w postaci rękawicy oraz otwartego kodu odbiornika zostanie przekazany dr inż. Andrzejowi Wołczowskiemu (wyłączając moduł odbiornika z racji wkładu własnego użytego do jego budowy).

Z racji realizacji projektu w celach dydaktycznych kod projektu będzie otwarty. Prawa do dystrybucji rękawiczki posiadają wszyscy członkowie grupy, pod warunkiem równomiernego podziału zysków na wszystkich członków.

Wszelkie konflikty zaistniałe podczas realizacji projektu będą rozwiązywane w sposób demokratyczny z możliwością zawetowania wyniku głosowania przez koordynatora.

Koordynator zespołu ma możliwość ocenienia pracy każdego z członków poprzez uwzględnienie jego zaangażowania. Zadaniem koordynatora jest przydzielenie poszczególnych zadań, weryfikacja jakości, postępu oraz możliwość modyfikacji w razie takiej konieczności. Rolą koordynatora jest także motywacja, wsparcie mentalne i merytoryczne oraz rozwijanie umiejętności miękkich wśród reszty grupy projektowej.

Postępy prac będą monitorowane poprzez odgórny nadzór koordynatora. Regularne spotkania będą się odbywać podczas zajęć projektowych według kalendarza PWr. Wszelkie dokumenty robocze będą składowane na grupie znajdującej się na portalu Facebook.com, kod źródłowy oprogramowania będzie umieszczany przez serwis github.com. Komunikacja zdalna będzie się odbywać poprzez powyższe serwisy internetowe. 
