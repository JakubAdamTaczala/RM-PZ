\section{Problem projektu}
\label{sec:problem_projektu} %definicja etykiety rozdzialu

\subsection{Nazwa projektu} % deklaracja podrozdziału
\label{subsec:nazwa_projektu}         % deklaracja etykiety podrozdziału
Rękawica sensoryczna – wersja „komercyjna”.
\subsection{Opis ogólny} 
\label{subsec:opis_ogolny} 
 
 
Głównym celem projektu jest budowa rękawicy wyposażonej w czujniki zgięcia i nacisku. Po analizie prototypów rękawic dostępnych w laboratorium 06 bud. C-3 ustalono, że obecne rozwiązania nie zapewniają stabilności (technologia wykonania jest nieestetyczna, co może być przyczyną licznych uszkodzeń w trakcie użytkowania). Pojawiła się więc potrzeba zaprojektowania i stworzenia gotowego, mobilnego i wygodnego dla użytkownika produktu.\\ Z tego powodu powstała współpraca dwóch grup projektowych z kursu Roboty Mobilne, które, wykorzystując jedną bazę (cienką i przylegającą rękawiczkę termoaktywną), będą odpowiedzialne za realizację dwóch osobnych zadań.

Grupa pierwsza w składzie: Gottschling Martyna, Mioduszewski Sylwester, Taczała Jakub odpowiedzialna jest za obsługę czujników nacisku wraz z nadajnikiem danych (z wykorzystaniem modułu radiowego).\\
Grupa druga w składzie: Gałązka Rafał, Gogola Piotr, Grącki Wiktor odpowiedzialna jest za obsługę czujników zgięcia wraz z odbiornikiem danych radiowych, który to będzie łączył się z komputerem.

Wykorzystanie modułu radiowego pozwoli użytkownikowi uzyskać mobilność (brak ograniczeń wynikających z podłączonych przewodów zasilania i transmisji).
\subsection{Zastosowanie} 
\label{subsec:zastosowanie}        
Gotowy produkt może zostać użyty podczas rehabilitacji osób sparaliżowanych (medycyna) oraz do modelowania zachowania dłoni (robotyka).
\subsection{Spodziewane wyniki pracy} 
\label{subsec:spodziewane_wyniki_pracy}         
Zakładanym efektem końcowym jest rękawica wytrzymująca użytkowanie zgodne z dokumentacją, wygodna w użyciu.
\subsection{Sposób upowszechniania} 
\label{subsec:sposob_upowszechniania}        
Etapy realizacji projektu oraz efekt końcowy wraz z użytym oprogramowaniem 
i dokumentacjami będą dostępne na stronie:\\ www.diablo.ict.pwr.wroc.pl/\~{}wgracki.
